\chapter{Details of the different input and output signals of each components of the architecture} \label{appendix:sig}
We detail here the different input and output signals of each component of the overall architecture described in Section \ref{sec:implementation}. Each Table details the signals of one component.
%
\begin{table}[H]
    \begin{tabular}{c|c|c}
        Signals & Input/Output & Description \\
        %
        \hline \hline
        clk, rst & Input & Clock and Resest signals \\
        %
        \hline
        inf\_conv & Input & convolution information of the layer\\
        %
        \hline
        start & Input & \makecell{Tells the main controller that it can perform \\ the bottleneck convolution} \\
        %
        \hline
        \makecell{$s_{dma}$ \\$s_{c11}$\\$s_{dsc}$} & Output & \makecell{Starting signals telling that the corresponding \\ component can start its operation}\\
        %
        \hline
        finish & Output & \makecell{Enabled by the main controller \\ indicating the output \acrshort{fm} has been produced}\\
        %
        \hline
        \makecell{$f_{dma}$ \\$f_{c11}$\\$f_{dsc}$} & Input & \makecell{Finishing signals telling that the corresponding \\ component has accomplished its operation}\\
        %
        \hline
        \makecell{$dma_{op}$\\$dma_{info_1}$\\$dma_{info-mem_1}$\\$dma_{info_2}$\\$dma_{info-mem_2}$} & Output & \makecell{Since the \acrshort{dma} can handle multiple operations,\\ theses signals specifies which operations to perform \\ and which tile and its external memory address to fetch}\\
        %
        \hline
        first\_par & Output & \makecell{Control signal telling the \acrshort{dsc} \acrshort{pe} \\ that loads its first fetching group \\ Therefore, it does not reads \\ the data in the output \acrshort{fm} buffer} \\
        \hline \hline
    \end{tabular}
    \caption{Input and output signals of the main controller}
    \label{tab:mc_sig}
\end{table}
%
\begin{table}[H]
    \begin{tabular}{c|c|c}
        Signals & Input/Output & Description \\
        %
        \hline \hline
        clk, rst & Input & Clock and Resest signals \\
        %
        \hline
        inf\_conv & Output & \makecell{Convolution information of the layer \\ fetched from the external memory}\\
        %
        \hline
        \makecell{$s_op$ \\ $op$} & Input & \makecell{Starting signals telling the \acrshort{dma} that it can perform \\ the operation referenced by $op$} \\
        %
        \hline
        \makecell{$r_{valid\_extmem}$ \\ $data_{extmem}$} & Input & \makecell{If enabled, the signal $data_{extmem}$ corresponds \\ to a data fetched from external memory} \\
        %
        \hline
        \makecell{$r_{request\_extmem}$ \\ $addr_{extmem}$} & Output & \makecell{If enablded, requesting a data transfer \\ at address $addr_{extmem}$}\\
        %
        \hline
        $e_{op}$ & Output & \makecell{Enabled by the \acrshort{dma} \\ indicating the transaction is done}\\
        %
        \hline
        \makecell{$w_{fmi}$ \\$w_{kex}$\\$w_{kpw}$\\$w_{kdw}$\\$w_{ext}$\\$w_{data}$} & Output & \makecell{Write signals, enabled when \\ we write $w_{data}$ into the corresponding memory}\\
        %
        \hline
        $addr_{ram}$ & Output & \makecell{When reading FMO buffer, \\ indicates the address of the data}\\
        %
        \hline
        $ram_{data-i}$ & Input & \makecell{Data reads at address $addr_{ram}$ \\ in the FMO buffer}\\
        %
        \hline
        $r_{fmo-buff}$ & Output & \makecell{Enabled by the \acrshort{dma} when it reads\\ data from the FMO buffer}\\
        %
        \hline
        \makecell{$tx_{i}$\\$x_{mem-i}$\\$ty_{i}$\\$ty_{mem-i}$} & Input & \makecell{Theses signals specifies which tile \\ and its external memory address to fetc \\ (the data type is specified by $op$)}\\
        \hline \hline
    \end{tabular}
    \caption{Input and output signals of the \acrshort{dma}}
    \label{tab:dma_sig}
\end{table}
\newpage
