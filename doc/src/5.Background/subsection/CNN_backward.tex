\subsection{Back-propagation} \label{subs:trainbackward}
According to \textcite{ruder_overview_2017}, gradient descient optimization algorithms are the most popular to perform optimizations on \acrshort{nn}. These algorithms derive from the idea of gradient descent. As said at previous section, gradient descent is a way to minimize the loss function, parametrized by the model's parameters (weight). We can therefore describe gradient descent algorithm using equation \eqref{eq:gd}, where $\eta$ is the learning rate. If we update the weight in the opposite direction of the gradient, we can reach a local mininum.
%
\begin{equation}
    \boldsymbol{w} = \boldsymbol{w} - \eta \frac{ \partial L( \boldsymbol{x}, \boldsymbol{w} ) }{\partial \boldsymbol{w}}
    \label{eq:gd}
\end{equation}

\textbf{The original gradient descent} or \textbf{batch gradient descent }computes the gradient using the whole dataset. We see how to compute the gradient on equation \eqref{eq:gd-grad}. This might be improssible to do in practice if the dataset is too large. Variations have then be proposed to made the gradient descient to be practical.
%
\begin{equation}
    \frac{ \partial L( \boldsymbol{x}, \boldsymbol{w} ) }{\partial \boldsymbol{w}} = \frac{1}{Nin} \sum^{Nin}_{i = 0} \frac{ \partial L( x_i, \boldsymbol{w} ) }{\partial \boldsymbol{w}}
    \label{eq:gd-grad}
\end{equation}

\textbf{Stochastic gradient descent}, instead of using the entire dataset, performs the gradient descent algorithm on one sample at a time. We see how to compute the gradient on equation \eqref{eq:sgd-grad}. It avoids redundant computation of the batch gradient descent. It learns faster and can reach better local minima, however it is complicated to find the global mininum.
%
\begin{equation}
    \frac{ \partial L( \boldsymbol{x}, \boldsymbol{w} ) }{\partial \boldsymbol{w}} = \frac{ \partial L( x_i, \boldsymbol{w} ) }{\partial \boldsymbol{w}}
    \label{eq:sgd-grad}
\end{equation}

\textbf{Mini-batch gradient descent} is a trade-off between the two approaches. It performs an update of the weight for every mini-batch of $N$ training examples. It has better convergence  by reducing the variance and has less computation than batch gradient descent.
%
\begin{equation}
    \frac{ \partial L( \boldsymbol{x}, \boldsymbol{w} ) }{\partial \boldsymbol{w}} = \frac{1}{N} \sum^{N < Nin}_{i = 0} \frac{ \partial L( x_i, \boldsymbol{w} ) }{\partial \boldsymbol{w}}
    \label{eq:bgd-grad}
\end{equation}
