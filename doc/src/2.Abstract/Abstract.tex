\topskip0pt
\vspace*{\fill}
\begin{center}
    \huge{\textsc{Abstract}}
    \end{center}
    
    CNN are powerfull models used in image classification, speech recognition, medical image analysis, etc. However, those models requires a large number of parameters and operations to be executed. Therfore, GPUs are the dominant platform to implement a CNN thanks to their huge computational resources and memory. However, such platforms are power hungry and limit the implementation of CNN on mobile and embedded devices. A solution would be to use, rather than GPU, FPGA which are more energy efficient. However, the FPGA does not require enough resources to implement to most performing models. Therfore, this thesis aims at combining the advantages of two optimizations, pruning and depthwise separable convolution, to further reduce the number of parameters and operations required by a CNN. The performance of this new pruning scheme is studied by implementing such model on FPGA.
\vspace*{\fill}
\afterpage{\blankpage}
\newpage
